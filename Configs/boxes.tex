% Tcolorboxes
\makeatother
\usepackage{thmtools}
\mdfsetup{skipabove=1em,skipbelow=1em}

\theoremstyle{definition}
\declaretheoremstyle[
    headfont=\bfseries\sffamily\color{black!70!black}, bodyfont=\normalfont,
    mdframed={
        linewidth=2pt,
        rightline=true, topline=false, bottomline=false,
        linecolor=black, backgroundcolor=black!3!white,
    }
]{thmbox}

\declaretheoremstyle[
    headfont=\bfseries\sffamily\color{black!3}, bodyfont=\normalfont,
    mdframed={
        linewidth=2pt,
        rightline=false, topline=false, bottomline=false,
        linecolor=black, backgroundcolor=black!3!white,align=center,
    }
]{thmtikz}


\declaretheoremstyle[
    headfont=\bfseries\sffamily\color{black!70!black}, bodyfont=\normalfont,
    numbered=no,
    mdframed={
        linewidth=0pt,
        rightline=false, topline=false, bottomline=false,
        linecolor=black, backgroundcolor=black!2!white,
    },
    qed=\qedsymbol
]{thmproofbox}

\declaretheorem[numberwithin=chapter,style=thmbox, name=Definición]{definition}
\declaretheorem[sibling=definition,style=thmbox, numbered=no, name=Ejemplo]{eg}
\declaretheorem[sibling=definition,style=thmbox, name=Proposición]{prop}
\declaretheorem[sibling=definition,style=thmbox, name=Teorema, numbered=yes]{theorem}
\declaretheorem[sibling=definition,style=thmbox, name=Lema]{lemma}
\declaretheorem[sibling=definition,style=thmbox, name=Corolario]{corollary}

\declaretheorem[style=thmproofbox, name=Demostración]{replacementproof}
\renewenvironment{proof}[1][\proofname]{\vspace{-10pt}\begin{replacementproof}}{\end{replacementproof}}

\declaretheorem[style=thmbox, numbered=no, name=Nota]{note}
\declaretheorem[style=thmbox, numbered=no, ]{temp}
\declaretheorem[style=thmtikz, numbered=no, name=.]{tikznt}

\newcommand{\bb}[1]{\mathbb{#1}}
